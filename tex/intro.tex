The demand for cybersecurity professionals has been increasing and universities are failing to keep up \cite{workforce, hackers_wanted}. To address this demand, universities need to better educate students in cybersecurity. Despite this need, there is currently little research related to cybersecurity education and no comprehensive assessments to assess cybersecurity. In order to address the lack of assessments, we began the \gls{cat} project to develop tools to assess cybersecurity education programs. The first tool developed by the \gls{cat} project is the \gls{cci} that evaluates the basics of cybersecurity. This \DocTitle evaluates the \gls{cci} after an initial pilot test. We use the initial pilot results to explore potential insights into students' understanding and recommend modifications to the assessment. 

The \gls{cci} is a 25 multiple-choice-item assessment. The assessment covers the core concepts of cybersecurity as defined in a Delphi process \cite{delphi}. After developing the items, an expert panel rated and reviewed each item before giving the assessment to students. This expert feedback ensured items were clear and relevant. The assessment was then given to 142 students from multiple universities in a pilot trial. The results of the pilot trial were analyzed using psychometrics to determine validity and suggest modifications to the \gls{cci}. 

A valid assessment can be used to draw a reasonable inference of student's knowledge \cite{douglas_purzer}. The validity of the assessment is established by set of evidence and arguments that argue whether the assessment can be appropriately used to draw this inference. To establish the validity of our assessment, we are following the design and evaluation framework recommended by the National Research Council \cite{libarkin,knowing_what_students_know}.


In this \NoSpaceDocTitle, we review the background and previous work, define our methods for the expert panel and pilot trial, and present and analyze the results using \gls{ctt}. We use \gls{ctt} as the testing paradigm to analyze the validity and reliability of the assessment. We then look at items within the assessment that do not meet the accepted criteria and detail how we will go about fixing those items. To our knowledge, we are the first to create and administer a \gls{cilabel} for cybersecurity. 