\section{Motivation}

Computers are becoming ubiquitous: they are used in diverse contexts including medical equipment, cars, and appliances. Due to this ubiquity, both cybersecurity experts and non-experts using these tools need to understand the core concepts of cybersecurity. For example, after a string of ransomware attacks targeted hospitals, Wirth called for healthcare technology management professionals to undertake cybersecurity training \cite{htm, ransomware}. The number of fields requiring basic cybersecurity concepts will likely continue to rise as attackers' targets expand.

Because the industries being affected by cyberattacks are expanding, security professionals are in high demand. This demand cannot be met with the current levels of cybersecurity education. Libicki et al. \cite{hackers_wanted} go as far as to say ``the shortage of cybersecurity experts in the federal government is serious to the point of being a national security threat to the United States." Despite the importance and ubiquity of cybersecurity, there is little research on how to teach cybersecurity effectively. Creating a valid and broadly used conceptual instrument for cybersecurity is a vital resource for supporting rigorous research on the efficacy of various teaching methods for cybersecurity education. Unfortunately, no such validated research instruments exist to assess students' conceptual knowledge of cybersecurity. 

Sherman et al. began the \gls{cat} project to meet this need for validated research instruments to assess the effectiveness of cybersecurity education \cite{delphi, misconceptions, scenarios, jcerp, status, hackathon}. The \gls{cat} Project is developing two \glspl{cilabel} to evaluate how well teaching practices help students learn core cybersecurity concepts: the \gls{cci} and \gls{cca}. The \gls{cci} assesses how well a student has learned the basic concepts of cybersecurity after one cybersecurity course. The \gls{cca} assesses how well a student has learned cybersecurity concepts after completing a full cybersecurity curriculum.

\section{Validity and Concept Inventories}

\glspl{cilabel} have been used to show that students regrettably succeed in traditional assessments through fact memorization rather than conceptual understanding \cite{hake, fci, litzinger}. With a deeper conceptual understanding, students learn more efficiently in the future and transfer their knowledge across contexts \cite{litzinger}. \glspl{cilabel} have been effectively used to promote the adoption of evidence-based teaching practices across STEM that are conducive to students developing a deeper conceptual understanding \cite{hake, fci, ci_progress}.

A \gls{cilabel} can be powerful and useful only if it is deemed as a valid instrument by the education community that will use the instrument. A valid \gls{cilabel} effectively evaluates targeted concepts and can be used to draw a reasonable inference of student knowledge \cite{douglas_purzer}. The validity of the instrument is established by a set of evidence and arguments about whether the instrument can be appropriately used to draw these inferences. To establish the validity of our instrument, we are following the design and evaluation framework recommended by the National Research Council \cite{libarkin,knowing_what_students_know}. 

\section{Outline of Thesis}

In this \NoSpaceDocTitle, we review the development process of the \gls{cci} and how that process compares to the development of other \glspl{cilabel}. We then describe the framework we use to evaluate whether the \gls{cci} can be used validly to assess student knowledge of cybersecurity concepts. We then describe the research methods for the expert panel review and pilot test with students. We analyze the results of this pilot test using \gls{ctt}. We then discuss these findings to identify the strengths of the \gls{cci} and to recommend future improvements for the \gls{cci}.