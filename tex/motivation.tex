Computers are becoming ubiquitous: they are used in diverse contexts like medical equipment, cars, and appliances. Due to this ubiquity, the core concepts of cybersecurity need to be understood not only by cybersecurity experts but by non-experts who work with these tools. An example of attackers targeting these new domains is the ransomware attacks that targeted hospitals in 2017-2018 \cite{ransomware}. After these attacks, there were calls for healthcare technology professionals to require cybersecurity training \cite{htm}. The number of fields requiring basic cybersecurity concepts will likely continue to rise as attackers' targets expand.

Since the industries being affected by cyberattacks are expanding, security professionals are in high demand. This demand cannot be met with the current levels of cybersecurity education. Libicki et al. \cite{hackers_wanted} go as far as to say ``the shortage of cybersecurity experts in the federal government is serious to the point of being a national security threat to the United States." Despite the importance and ubiquity of cybersecurity, there is little research on how to teach cybersecurity effectively, and there are no validated assessments. 


The lack of research in cybersecurity research led us to develop two assessments of students' conceptual understanding of cybersecurity as part of the \gls{cat} project \cite{workforce}. The \gls{cci} covers the basic concepts of cybersecurity after one cybersecurity class. The \gls{cca} covers a full cybersecurity curriculum required for one to work as a professional. This \DocTitle covers the evaluation of the \gls{cci}. We intend to use the \gls{cci} to improve course design and to assess students' understanding.


We chose a \gls{cilabel} as the format of the evaluation assessment because \glspl{cilabel} have a long history of being used in \gls{stem} fields including electromagnetics, signals and systems, and circuits \cite{ci_progress}. \glspl{cilabel} have been used to show that students succeed in traditional assessments through fact memorization rather than conceptual knowledge \cite{hake}, \cite{fci}, \cite{litzinger}. With deeper conceptual knowledge, students learn more efficiently in the future and transfer the knowledge across contexts \cite{litzinger}. The effectiveness of conceptual learning has led to the development and adoption of pedagogies that support deep conceptual learning \cite{dlci}. The original concept inventory, the \gls{fci}, led to the adoption of active learning pedagogies that were shown to be effective \cite{hake}, \cite{fci}, \cite{evans}. We hope to develop similarly influential pedagogies for cybersecurity.



%\iffalse
%\textbf{Subject: Discuss the Force Concept Inventory (FCI).}

%The original \gls{cilabel} was the \gls{fci} developed for introductory physics courses in the 1990's \cite{fci}. The \gls{fci} was assessmental in introductory physics instruction and is the blueprint of modern \glspl{cilabel}. Hestenes et al. \cite{fci} administered the assessment to over 2000 high-school and university students. The data obtained from this trial had remarkable results in terms of reliability and validity.\gls{fci}  has supported the effectiveness and adoption of active learning pedagogies in physics education \cite{hake}.

%\fi
