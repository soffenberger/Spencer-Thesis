Parekh et al. \cite{delphi} began the \gls{cat} Project development by identifying the core concepts of cybersecurity using a Delphi process. A Delphi process is a rigorous and structured method for creating consensus among experts about potentially contentious issues, such as what subset of concepts should be included on the \gls{cci} \cite{original_delphi}. This process identified five concepts to include in the \gls{cci} seen in Table \ref{tab:topics} \cite{delphi}. From these concepts, Sherman et al. \cite{scenarios} developed cybersecurity scenarios that require students to understand these concepts. For example, a scenario that covers concept \gls{c} involves a hypothetical government facility where the defenses and biometric authentication methods are defined. This scenario allows for questions on potential attacks that exploit the defenses described in this scenario.

Using these scenarios, Scheponik et al. \cite{misconceptions} performed think-aloud interviews to discover students' misconceptions and problematic reasoning \cite{jcerp}. Example forms of problematic reasoning include students' beliefs that encryption protects against most any cybersecurity threat and the belief that cybersecurity threats come only from outside an organization.

Using what findings from these interviews, we created \gls{cci} items using the same scenarios. Each \gls{cci} item consists of a scenario, a stem (i.e., a question about the scenario), and five answers. The wrong answers (distractors) were created based on the interview findings.