%The pilot trial for the \gls{cci} included 142 students from 6 universities. The \gls{cci} had very good reliability and support from the expert panel. There suggested modifications include changing items to make them easier and to re-evaluate the individual concepts after scaling up to more students. The next step is to make the proper modifications and then administering the \gls{cci} to more students in a full trial. 

%Write 

%The expert review and pilot administration of the \gls{cci} revealed WHAT?
%- testing core cybersecurity conceptual knowledge
%- is reliably testing that knowledge
%- could be used at this point to measure knowledge with the caveat that scores will be low and will lack discriminatory power. By making the cci easier, we will be able to create an assessment that should be broadly applicable and provide useful measurements of a broad range of cybersecurity students.

The expert review and pilot testing of the \gls{cci} revealed the \gls{cci} reliably tests students' knowledge of cybersecurity. At this point, the \gls{cci} could be used as an evaluation instrument but the scores would be low, reducing the discriminatory power of the assessment. By making the \gls{cci} easier, we will be able to create an assessment that should be broadly applicable and provide useful measurements of a broad range of cybersecurity students. Further research will cover the modifications of the items and testing with more students. 