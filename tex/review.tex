ENE 695 005/006 WRITING GUIDELINES Spring 2019

M. C. Loui

I will comment on your draft papers to guide your revisions toward the next version. In the comments, I will focus on the content, clarity, and style. I will not edit your writing to correct errors of grammar, word choice, spelling, and punctuation—you will be responsible for these errors. As you revise your drafts, you should not merely correct individual errors and amend individual words; you should alter the overall organization and rewrite entire paragraphs to improve their coherence and flow (cohesion).

This sheet outlines basic expectations that all writing in this course should meet. If your paper does not satisfy the expectations on Walvoord’s Big 8 list, it will be returned to you without a grade.

Unless otherwise specified, every paper should have one-inch margins, and every page should be numbered. The main text should be double-spaced with a 12-point font size. Each paper should be submitted in Blackboard as document in PDF, DOC, or DOCX format.

Walvoord’s Big 8 List

1. Does the sentence make sense? Are the words accurate?

2. Are there sentence fragments, run-ons, or comma splices?

3. Has the writer avoided using a comma to separate a subject from a verb or a verb from an object?

4. Do subjects and verbs agree?

5. Are verb tenses appropriate and consistent?

6. Do pronouns match their antecedents?

7. Are apostrophes in the right places?

8. Are there misspellings or misused words like their/there, sit/set, or lie/lay?

Walvoord, B. E. (2014). Assessing and improving student writing in college: A guide for institutions, general education, departments, and classrooms (p. 80). San Francisco: Jossey-Bass.

Professor Loui’s Pet Peeves

Unclear referents for pronouns

Example: “As a girl, I liked to play with computers, which was not supported by my parents. Fortunately, a teacher in high school saw that I was good at it and advised me to major in engineering. This was more difficult than I had imagined.”

Revision: “As a girl, I liked to play with computers, but my parents didn’t support my interests. Fortunately, a teacher in high school noticed my talent with computers and advised me to major in engineering—advice that was more difficult to follow than I had imagined.” [Paraphrased from an example by John C. Bean.]

Dangling participle

Example: “Typing on my computer, the alarm rang.”

Revision: “As I typed on my computer, the alarm rang.”

Fused participle

Example: “Do you mind me playing the piano?”

Revision: “Do you mind my playing the piano?”

Incoherence

Example: “American corporations hire engineers to develop new weapons. The public demands increasingly sophisticated weapons. Many engineers have moral reservations about participating in weapons development. Weapons of mass destruction raise particular concerns because they kill indiscriminately. Some engineers avoid careers in weapons development.”

Revision: “Although engineers are hired by American corporations to fulfill the public’s demands for increasingly sophisticated weapons, many engineers have moral reservations about developing new weapons, particularly weapons of mass destruction, which kill indiscriminately. Because of these moral concerns, some engineers avoid careers in weapons development.”

Guidelines for Gender-Inclusive Usage

Use gender-inclusive language in your writing, unless the gender of the specific person is known. Avoid cumbersome devices such as slashes (he/she, his/her) and the repetitive use of the conjunction or (he or she, his or her). Here are some basic methods:

1. Use the noun

2. Use the plural

3. Delete the pronoun

4. Use an article or conjunction: a, an, the, but, and

5. Use the following relative pronouns: who, whom, whose

6. Use the passive

7. Rewrite the sentence

Example #1: “The final judge of a young man’s gradual advance is as much or more his

employer as his professional colleagues.” [Layton]

Revision: “The final judge of the engineer’s gradual advance is as much or more the

employer as professional colleagues.” (1, 3, 4)

Example #2: “An engineer should prepare his drawings carefully.”

Revisions: “Engineers should prepare their drawings carefully.” (2)

“Drawings should be prepared carefully.” (6)

Example #3: “If a consulting engineer has worked for a competitor, then he should disclose the possible conflict of interest to his potential client.”

Revision: “A consulting engineer who has worked for a competitor should disclose the possible conflict of interest to a potential client.” (5, 4)