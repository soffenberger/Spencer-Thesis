The \gls{cci} has 15 scenarios, each of which describes a security problem. Each scenario of the \gls{cci} presents several question stems. Each stem is followed by five possible answer choices. One answer is correct. The other, incorrect, answers are called \textit{distractors}. Each multiple-choice question is called an \textit{item}. The current \gls{cci} has 25 items.

%\textbf{Subject: Discuss the development.}

%The CCI scenarios are based on scenarios developed by Sherman et al. \cite{scenarios}. To develop the CCI scenarios and questions, Sherman et al. held a "hackathon" with 17 experts from industry and academia. Sherman et al. collaborated with experts to develop high-quality items that are relevant to the five core security concepts. Sherman et al. performed think-aloud interviews with students about these scenarios. From the think-aloud interviews, Scheponik et al. developed common student misconceptions \cite{misconceptions}. These misconceptions were used to form the distractors of each item.

\section{Expert Panel}

We compiled the items developed and formed the \textit{initial \gls{cci}}.  The initial \gls{cci} contained 32 items. To further establish the validity of the \gls{cci}, we gave these items to an expert panel for review. The expert panel consisted of 11 professors with backgrounds in cybersecurity and 1 cybersecurity professional. These professors were from diverse universities across the country. The cybersecurity professional worked as a consultant in cybersecurity. The experts each received the initial \gls{cci} in the form of an online exam. The online exam contained each of the 32 items. Additionally, experts were asked to rank each item on the scale Accept, Accept with Minor Revisions, Accept with Major Revisions, and Reject. Included in each item was a comment box for any written feedback from experts. Experts were instructed to leave a ranking, write any feedback, and answer the question. After answering, experts were shown the correct answer. The experts were then given the option to provide comments on the correct answer. 


To incorporate the feedback, we concluded taking expert feedback before giving it to students. The expert feedback consisted of each expert's comments and rankings. The experts' comments mainly covered issues with clarity including problems with wording and implicit assumptions. We fixed wording issues by changing the item to address the complaint and checking back with experts for approval. We also stated any assumptions that experts thought were unclear. For some items, the experts disagreed with the content or the correct answer. When the experts disagreed, we omitted that item from the \gls{cci}. After we removed these items, the experts' reviews were used to rank the remaining items. The highest-ranked items were incorporated into the current \gls{cci}, demonstrating that the \gls{cci} should be validly applicable to students beyond our institutions.

\iflong

During the expert reviews, we continued developing items for the \gls{cci}. These items were intended to be used as alternates. One alternate item, Q25, was not reviewed by experts but was included. We thought this item evaluated the core concept \textit{\gls{c}} better than those that were reviewed. This item and those with the best reviews form the current 25-item \gls{cci}.

\fi


\section{Current \gls{cci}}

We selected items with a range of difficulties based on our best estimation. The breakdown of questions is: 6 easy, 16 medium, and 3 hard. The actual performance of students will likely differ from our estimations. Each item covers one of the five major concepts shown in Table \ref{tab:topics}. \iflong The items are shown in Table \ref{tab:final_question_breakdown}. This table shows the scenario, name, concept, topic of each item in the current \gls{cci}.\fi 



\glsreset{c}
\glsreset{v}
\glsreset{d}
\glsreset{g}
\glsreset{t}


\iflong
\begin{table}[!h]
\centering
\caption{Five Core Concepts of Cybersecurity}
\scalebox{.9}{
\begin{tabular}{c}
\toprule
  \gls{v}\\
  \gls{c}\\
  \gls{d}\\
  \gls{g}\\
  \gls{t}\\
\bottomrule
\end{tabular}
}
\label{tab:topics}
\end{table}
\fi
\ifshort
\begin{table}[!h]
\centering
\caption{Five Core Concepts of Cybersecurity}
\scalebox{.7}{
\begin{tabular}{c}
\toprule
  \gls{v}\\
  \gls{c}\\
  \gls{d}\\
  \gls{g}\\
  \gls{t}\\
\bottomrule
\end{tabular}
}
\label{tab:topics}
\end{table}
\fi




\FloatBarrier






\section{Pilot Trial}

The goal of the pilot trial was to administer the current \gls{cci} to a small group of 100-200 students and use the results to suggest modifications to the assessment. The pilot trial was completed in December 2018 by 142 students from 6 universities.

Professors at each university had the option of administering a paper version or online version of the \gls{cci}. Both versions included instructions at the beginning of the exam. \iflong The instructions can be seen in Appendix B. \fi Both versions also had all of the items belonging to the same scenario appear sequentially. The distractors, scenarios, and questions were identical in both versions. 

If the professors decided to administer the paper version, the professor proctored this version as an exam. To proctor the exam, a professor allocated 50 minutes for students to take the exam in class. Students then completed the exam to the best of their abilities. The professor collected the exam papers and sent them to us. We then recorded each student's response to every item. 

If the professor decided to administer the online version, students were provided a link to the exam. The online exam differed from the paper version in three ways. First, the online version had a random ordering of distractors. Second, items that shared a scenario were randomly ordered within that scenario. For example, if Q1 and Q2 are the two items in the one scenario, Q1 can appear before or after Q2. However, they would always appear together. Third, there was no hard time limit. Students were told to spend 50 minutes on the exam, but this was not strictly enforced. The student completed the exam and then selected a submit button, thus saving the responses.


\FloatBarrier
\section{Pilot Demographics}

The universities included in the pilot trial have diverse locations and populations. Universities A and D are large Midwestern public universities and have over 40 thousand students enrolled. University E is a large public university in the South with over 40 thousand students enrolled. Universities B, C, F are smaller universities in the Midwest and East with 10k or fewer students enrolled.

The demographics of the study including institution and response rate are in Table \ref{tab:student_breakdown}. University A was the only group given the assessment in paper format. The professors at the other universities sent out the link to the assessment to the students in the course. With one exception, at University D, the assessment was to the six members of this club who are taking this course. This club sent out the link to members who were taking the first cybersecurity course. 



\iffalse
\begin{table}[!htbp]
\centering
\begin{tabular}{cc}
    \toprule
    \textbf{University/Organization} & \textbf{Number of Experts}\\
    \midrule
    \textit{University of Utah} & 1\\
    \textit{Capitol Tech} & 1\\
    \textit{Texas A \& M} & 1\\
    \textit{University of Southern California} & 1\\
    \textit{Association for Computing Machinery (ACM)} & 1\\
    \textit{University A} & 2\\
    \textit{Depaul University} & 1\\
    \textit{Michigan Tech University} & 1\\
    \textit{University G} & 1\\
    \textit{Montgomery College} & 1\\
    \textit{University B} & 1\\
    \textif{No University Specfified} & 4\\
    \midrule
    \textit{Total} & 12\\
    \bottomrule
\end{tabular}   
\caption{Breakdown of Professors By University/Organization}
\label{tab:proffesor_breakdown}
\end{table}
\fi

\iflong
\begin{table}[!htbp]
\caption{Breakdown of Students by University}
\centering
\scalebox{.6}{
\begin{tabular}{cS[table-number-alignment = center]S[table-number-alignment = center]S[table-number-alignment = center]}
    \toprule
    \textbf{University} & \textbf{Number of Students} & \textbf{Potential Number of Students} & \textbf{Response Rate (\%)}\\
    \midrule
    \textit{University A} & 91 & 120 & 76 \\
    \textit{University B} & 12 & 20 & 60 \\
    \textit{University C} & 1 & 12 & 16\\
    \textit{University D} & 6 & 6 & 100 \\   
    \textit{University E} & 17 & 50 & 34 \\
    %\textit{University F} & 6 & 40 & 15 \\
    \textit{University F} &	12 & 20 & 60\\
    \textit{No University Specified} & 3 &  & \\
    \midrule
    \textit{Total} & 142 & 228 & 62 \\
    \bottomrule
\end{tabular}   
}

\label{tab:student_breakdown}
\end{table}
\fi

\ifshort
\begin{table}[!htbp]
\centering
\scalebox{.6}{
\begin{tabular}{cc}
    \toprule
    \textbf{University} & \textbf{Number of Students}\\
    \midrule
    \textit{University A} & 91  \\
    \textit{University B} & 14  \\
    \textit{University C} & 1 \\
    \textit{University D} & 6  \\   
    \textit{University E} & 17 \\
    %\textit{University F} & 6 & 40 & 15 \\
    \textit{University F} &	12\\
    \textit{No University Specified} & 3 \\
    \midrule
    \textit{Total Number of Completed Exams} & 142\\
    \bottomrule
\end{tabular}   
}
\caption{Breakdown of Students By University}
\label{tab:student_breakdown}
\end{table}
\fi

\iffalse
\begin{table}[!htbp]
\centering
\begin{tabular}{cc}
    \toprule
    \textbf{Year} & \textbf{Number of Students}\\
    \midrule
    \textit{Sophomore} & 79 \\
    \textit{Junior} & 24 \\
    \textit{Senior} & 29 \\
    \textit{Graduate} & 11 \\   
    \bottomrule
\end{tabular}   
\caption{Breakdown of Students By Year}
\label{tab:student_year}
\end{table}
\fi