%\subsection{Concept Inventory Evaluation}

Because a test cannot be universally valid for every population or purpose, we need to carefully define the contexts, populations, and purposes for which the \gls{cci} is valid. The \gls{cci} is intended to measure the cybersecurity conceptual knowledge of students who have completed a first course in cybersecurity. Cybersecurity is taught to an increasingly wide range of stakeholders, such as policy makers, computer scientists, medical professionals, and business professionals. These courses vary in focus in depth, for different kinds of students. Because of this high variance, we have chosen to optimize the \gls{cci} for the largest population of computer scientists. While the \gls{cci} may provide useful insights about the conceptual knowledge of policy makers or others, our goal is to have the tool provide the most insight about computer science students. 

Once an assessment tool is created, it should be administered to its targeted demographic and be statistically evaluated. \glspl{cilabel} can be powerful instruments if they actually measure students' conceptual knowledge. A minority of \glspl{cilabel} have been scrutinized using measurement or test development theories to justify being a valid and reliable research instrument \cite{dlci}. Jorion et al. \cite{jorian} outline three basic criteria of a valid \gls{cilabel}: (1) the \gls{cilabel} indicates overall understanding of the concepts, (2) it indicates understanding of a specific concept, and (3) it indicates misconceptions or student errors.

 Jorion et al. recommended using a series of statistical tests to demonstrate whether a \gls{cilabel} meets these criteria. They recommended beginning analysis of \gls{cilabel} using \gls{ctt}. According to \gls{ctt}, an assessment instrument should minimize error. All of the instrument's items should test a single construct. Each item should be neither too hard nor too easy. Each item should provide a good estimate of the student's overall ability.
 
 %\gls{ctt} argues that an assessment tool should minimize error and possess items that all test a single construct, that are neither too hard nor too easy, and that each provide a good estimate of a student's overall ability. 